\documentclass[11pt]{article}
\title{Numerically Solving Newton's Second Law}
\date{}

\begin{document}
\maketitle

\section{Pre-notebook}
Work through these questions with your group before you move onto the IPython notebook.

\subsection{Solving Newton's Second Law for a mass on a spring}\label{Newtons}
Write down Newton's Second Law for a mass $m$, on a spring with spring constant $k$.
Initially assume that there is no friction in the system.
\begin{itemize}
\item Can you rewrite this in terms of derivatives of the position variable?
\item Solve this differential equation and find an expression for the position of the mass as a function of time.
\end{itemize}

\subsection{Conservation of Energy}
Given the solution you found in section~\ref{Newtons}, find the velocity of the mass as a function of time.
With the position and velocity expressions, check to make sure that the system conserves energy.

Okay, now go and start working through the notebook.

\section{Numerical Updates for Position and Velocity}
We want to model our system discretely so it can be solved on a computer.
\subsection{Written Updates}
Given the ``delta t'' versions of the acceleration and velocity equations, what are the updates for $v_{t+\Delta t}$ and $x_{t+\Delta t}$.
\subsection{Coded Updates}
We've filled in the update for $x$ and $t$ for you. Write the code to update $a$

\section{Problems with the Model}
If you made your $\Delta t$ larger, you might have noticed that the oscillations started to blow up in our numerical version.
\subsection{Conservation of Energy}
Can you show if energy is conserved given the updates for $x$ and $v$ that you derived earlier?
\subsection{Adding Drag}
What if we want to add drag to the model? Write down the updates for $x$ and $v$ again with a drag term included.



\end{document}
